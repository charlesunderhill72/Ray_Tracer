%
% project_paper.tex - LaTeX project presentation paper
%
% 28Jul21  Charles Underhill
%
\documentclass[12pt]{article}
\usepackage{fancyhdr}
\usepackage{graphicx}
\usepackage{color}
\usepackage{hyperref}
\usepackage{amsmath}

% Uncomment one of the following lines to use Times-Roman and Helvetica
% or Palatino instead of Computer Modern.
% \usepackage{txfonts}
% \usepackage[sc]{mathpazo}\linespread{1.05}\usepackage[scaled]{helvet}

\usepackage[hmargin=90bp,tmargin=108bp,bmargin=72bp,
            headheight=15bp,footskip=40bp]{geometry}
%%%%%%%%%%%%%%%%%%%%%%%%%%%%%%%%%%%%%%%%%%%%%%%%%%%%%%%%%%%%%%%%%%%%%%%%%%%%%%%

%
% custom definitions
%
\newcommand\thisis{3D Mandelbrot Set Ray Tracer}
\newcommand\theauthor{Charles~Underhill}

\newcommand\sfb{\sffamily\bfseries}

\newcommand\red[1]{\textcolor{red}{\sffamily\bfseries #1}}
%%%%%%%%%%%%%%%%%%%%%%%%%%%%%%%%%%%%%%%%%%%%%%%%%%%%%%%%%%%%%%%%%%%%%%%%%%%%%%%

%
% custom heading and footer
%
\fancypagestyle{firstpg}
   {
   \fancyhf{}%
   \cfoot{\sffamily\thepage}%
   \renewcommand\headrulewidth{0bp}
   }

\pagestyle{fancy}
\lhead{\sffamily \thisis}
\chead{}
\rhead{\sffamily \theauthor}

\lfoot{}
\cfoot{\sffamily\thepage}
\rfoot{}
%%%%%%%%%%%%%%%%%%%%%%%%%%%%%%%%%%%%%%%%%%%%%%%%%%%%%%%%%%%%%%%%%%%%%%%%%%%%%%%

\begin{document}
\thispagestyle{firstpg}

\noindent
{\sffamily\bfseries\huge \thisis}\\

\noindent
{\large\sffamily \theauthor}

\vspace*{20bp}

\indent
This project requires no additional hardware or software other 
than the standard Raspberry Pi setup for Physics 129L. 

\vspace*{10bp}
\indent
The project is a program that computes the Mandelbrot Set in a 
user specified region and renders a 3D image of the region with 
a ray tracing algorithm. The program starts by asking the user 
to enter in the coordinate range and a number of points along 
the x and y (real and imaginary) axes to compute the Mandelbrot 
set. The program requires that the same amount of points be 
computed along both axes so the user is only prompted once for 
N points in both x and y directions. 

\vspace*{10bp}

\indent
If the user inputs the previous values correctly, they will be 
prompted for a view angle and a light height. The view angle is 
the angle from the x-y plane that the screen which holds the 
projected image will be situated at. The angle must be in between 
0 and 90 degrees. The light height is the height above the x-y 
plane where the light is positioned in the scene. Any value can 
be input here but something between 0 and 2 is recommended for 
the scene to be visible. 

\vspace*{10bp}

\indent
From here the program computes the Mandelbrot Set and stores the 
stability of each point in the range as a number which will be 
interpreted as a height above ground. Using this array the program 
computes surface normals with an approximate formula in order to 
calculate the trajectory of light rays off of the surface. The 
function below can then be used to calculate light intensities.

   \[ I(\vec{p}) = \begin{cases}
      I_a & \cos(\theta) < 0 \\
      I_a + f_d\cos(\theta) + f_s * (\cos(\alpha))^b & \cos(\theta) \geq 0
   \end{cases}
   \]

\indent
The function holds weights for ambient, diffuse, specular, and 
shiny components of light. The parameter \(\theta \) refers to the 
reflection angle at each point on the surface and \(\alpha \) is 
the angle that the surface normal makes with the view vector, 
which is the vector from surface environment to the screen. These 
intensities are then projected onto the screen using the floating 
horizon method and the following function:

   \[ P(y, z) = \text{int}((N - 1)(y\sin(\phi) + z\cos(\phi)) \]

\vspace*{5bp}

The floating horizon method involves projecting one vertical line 
at a time and keeping track of the maximum y value projected. Any 
points projected above the horizon are considered not visible by 
the light and are set to black. The pixel values on the screen are 
saved in an array and are plotted with a colormap using PyPlot.
 
\vspace*{10bp}
\newpage
\indent
Here are some interesting results from using this program. To get 
these results the user simply has to input the coordinate ranges, 
view angle, and light heights specified by the figure captions.

%------------------------------------------------------------------------------
\begin{figure}[h]
\includegraphics[width=400bp, scale=5]{project_example4}
\vspace{-18bp}
\caption[]{\label{fig:4}\small
The full Mandelbrot Set using coordinates x0 = -2, x1 = 1, y0 = -1, 
y1 = 1. The view angle is set to 45 degrees and the light height is 
0.2.
}
\end{figure}
%------------------------------------------------------------------------------
As you can see in Fig.~\ref{fig:4}, lighting effects are visible on 
the Mandelbrot Set due to the nature of ray tracing techniques. To 
get this output the user must enter -2, 1, -1, 1, 300, 0.2, and 45 
in that order after running the program. The same instructions apply 
for the next pictures. (x0, x1) define the x range, and (y0, y1) 
define the y range. The number of data points in each of these pictures 
is 1000, but the user can enter 300 to get a picture of similar quality 
with far less computing time. 

%------------------------------------------------------------------------------
\begin{figure}[h]
\includegraphics[width=\textwidth, height=\textheight, keepaspectratio]{project_example3}
\vspace{-18bp}
\caption[]{\label{fig:3}\small
The coordinates are x0 = -0.80, x1 = -0.72, y0 = 0.09, and y1 = 0.15. 
View angle is 80 degrees and height is 0.95.
}
\end{figure}

\begin{figure}[h]
\includegraphics[width=\textwidth, height=\textheight, keepaspectratio]{project_example2}
\vspace{-18bp}
\caption[]{\label{fig:2}\small
The coordinates are x0 = -0.75, x1 = -0.737, y0 = -0.13, and y1 = -0.12.
View angle is 85 degrees and height is 0.1.
}
\end{figure}


\begin{figure}[h]
\includegraphics[width=\textwidth, height=\textheight, keepaspectratio]{project_example1}
\vspace{-18bp}
\caption[]{\label{fig:1}\small
The coordinates are x0 = -0.60, x1 = -0.52, y0 = -0.67, and y1 = -0.61.
View angle is 85 degrees and height is 0.1.
}
\end{figure}

\vspace{10bp}
\indent
As you can see from comparing Fig.~\ref{fig:3} and Fig.~\ref{fig:1}, 
changing the position of the light can have dramatic effects on the 
image produced by the program. The taller sections of the image appear 
much darker when the light is positioned higher like in Fig.~\ref{fig:3}. 
Additionally, comparing Fig.~\ref{fig:4} with the other figures, 
the difference in view angle is apparent with a higher view angle 
showing more of the scene but a lower view angle showing more contrast 
in height.

\end{document}
